\documentclass{rapport}
\usepackage[utf8]{inputenc}

\usepackage{pifont} % For symbols called by the \ding macro
\usepackage{url} % As its name suggests, for URLs...

\usetikzlibrary{positioning} % TikZ library for positioning nodes relative to others

\usepackage[colorlinks, citecolor=red!60!green, linkcolor=blue!60!green, urlcolor=magenta]{hyperref} % For clickable links. Options set link colors.

\usepackage{algorithm}
\usepackage{algo}
\usepackage{colorationSyntaxique}

% For a report in English
%\usepackage[english]{babel} % Uncomment for a report in English
%\englishTitlePage % Uncomment for an English title page

\pagestyle{fancy}
\renewcommand{\sectionmark}[1]{\markboth{\thesection.\ #1}{}}
\fancyfoot{}

\fancyhead[LE]{\textsl{\leftmark}}
\fancyhead[RE, LO]{\textbf{\thepage}}
\fancyhead[RO]{\textsl{\rightmark}}

\def\Latex{\LaTeX\xspace}
\def\etc{\textit{etc.}\xspace}

\title{Title of my awesome work}
\author{Me, Myself, I}
\supervisor{Toto, Tata, Tutu}
\date{Second semester of the year 2021-2022}

% \universityname{Côte d'Azur University} % Name of the university.
% \type{TER} % Type of document
% \formation{Master in Computer Science} % Name of the program

% See other possible options in the rapport.pdf document

\begin{document}

  \maketitle

  \begin{abstract}
    This document is an example for your future project, TER, or internship reports. It gives you some advice and guidelines for writing, a suggested outline for your TER report, and examples of \Latex usage. You can find many tutorials to learn how to write a document with \Latex, especially at \url{https://www.latex-project.org/help/documentation/} and of course on Stack Overflow \url{https://tex.stackexchange.com/}.

    This document was written by four hands, by \href{mailto:marie.pelleau@univ-cotedazur.fr}{Marie Pelleau} and \href{mailto:sid.touati@univ-cotedazur.fr}{Sid Touati}.
  \end{abstract}

  \clearpage
  \tableofcontents

  \clearpage

  %%%%%%%%%%%%%%%%%%%%%%%%%
  % Guidelines and Advice %
  %%%%%%%%%%%%%%%%%%%%%%%%%
  \part{Some Guidelines and Advice}

    \section{Introduction}
    
      You have reached the second cycle or are finishing your Master's degree. During your higher education, you have learned fundamental concepts in computer science, as well as technical hardware and software skills. Your studies or projects end with the writing of a final report, which will give an overview of your level and skills. It would be a shame to finish with a poorly written report. Especially if you are writing your internship report, which will remain in your academic file as the ultimate effort of your work. This document gives you some advice to improve the form of your report, both in terms of structure and writing. An impeccable report gives elegance to your mind.

    \section{The Editing Tool: \Latex Highly Recommended}

      In all disciplines in the so-called hard sciences (physics, mathematics, computer science, astronomy, electronics, chemistry, astrophysics, \etc), the essential tool for scientific writing remains \Latex. It is not only a matter of tradition and standard, but also for practical reasons: \Latex allows you to edit clear, clean, and long-lasting documents. It offers many possibilities to configure and manipulate scientific text, formulas, diagrams, and various other objects. A \Latex document is visually recognizable and gives immediate credibility to its author.

      Other tools like Word and OpenOffice are designed for general office use. They are nevertheless used for scientific writing in life sciences and humanities. These are software tools that allow you to see the final rendering of documents while writing, facilitate collaborative editing, version documents, track changes, \etc. They are intended for people who do not master advanced computer tools, which is of course not your case (at least we hope so). Tools like Word and similar do not allow you to create long-lasting documents like \Latex (Word software version changes often modify the visual rendering of the document) and their visual appearance does not have the same effect among scientists. A Word document in science is mainly used for office work, while \Latex is used to write articles, theses, books, and reports.

    \section{Structure and Outline of Your Report}

      Before writing your report, first think about its architecture (outline). It is the backbone of your document on which the different parts will be attached. Writing a report without initially thinking about its structure would result in an unbalanced document, and this would be felt when reading. The sections that must appear in all reports are: the abstract, the introduction, the state of the art (and therefore the bibliography), and the conclusion. To these standard sections, you will add your own sections related to your internship: problem description, proposed solution, experimentation, analysis and perspectives, \etc. It is highly recommended to give explicit names to your sections, so that the table of contents is clear and personalized. The following sections describe the expected content in each standard section.

      \subsection{The Abstract: Make People Want to Read Your Report}
        This is a paragraph intended to give an overview of your report: what field, what problem, what contribution. It is the equivalent of a movie teaser, in the sense that it should be clear enough for the reader to have an idea of the content and contribution, without giving all the important details, to make them want to read the report.

      \subsection{The Introduction: It's the Opening}
        You have spent several months working on a specific topic during your internship, or getting started on a research topic. The introduction section draws the general picture of the field in which you worked, its importance, the motivation behind your internship topic. Some fields are easier to describe because they have a direct impact on people's social and economic lives. Other fields are more complicated to motivate or explain; the introduction section is there to place your internship topic in the galaxies of sciences, branches, specialties, topics, \etc. It's like a star that you have to place on the sky map.

        At the end of your introduction, it would be good to briefly describe the different chapters (or sections) of your report, with the link that brings them together. Also, in your chapter text, it is good to briefly describe the different sections (or subsections) with their link. This will prevent the reader from seeing your report as a catalog containing a succession of sections without apparent connection.

      \subsection{The State of the Art: Your Synthesis Skills}
        In any internship or thesis work, you must spend time reading and studying what has been published elsewhere, related to your topic. What a shame to find that your topic has already been addressed by other research teams long before you, without you knowing! Bibliographic research is mandatory in any research work. Sometimes, several teams have worked exactly on the same topic but without realizing it because they used different technical terms, thus thinking the topics are different. That's why mathematical or formal abstraction of problems is sometimes necessary to take a step back and be able to communicate with other researchers in adjacent or distant fields.

        The state of the art section contains your description of the work related to your work. It is not just about filling space and copying/pasting summaries of articles you may have picked up here and there. It is a section that contains your own synthesis and critique of your readings. This section refers to several scientific documents listed at the end of your report, called the bibliography.

      \subsection{The Bibliography: To Be Chosen Carefully}
        The bibliography is the section at the end of the report that contains the list of all the documents you reference. The documents you reference must be chosen carefully, in the sense that you must list credible and accessible documents; not every source has its place in a scientific document. Thus, documents reviewed by specialist committees (articles published in scientific journals or conferences, theses) are more credible than press documents (articles in popular magazines, technical reports, blogs, websites). Moreover, when you reference a document, give all the details that allow the reader to find it precisely: authors' names and first names, document title, publisher, publication date, conference location, ISBN number if applicable, university name if it is a doctoral thesis, \etc.

        Also, the referenced document must be long-lasting, meaning it is unlikely to disappear in the coming years: for example, a link to an obscure web page is not a good bibliographic reference, as such a page can easily and quickly disappear, making your reference obsolete. The same goes for a document that has never been published or is present in no documentation center except your office. When writing your report, think of the reader who will print your report 5 or 10 years after you.

      \subsection{The Conclusion: The Section Everyone Will Read}
        Often, master's students are exhausted when they reach the end of the report to write the conclusion. As regularly observed, this section is unfortunately neglected, or the student simply recalls the content of their report. However, the conclusion section is the one everyone will read, as it is supposed to contain your last important message. Most readers probably did not understand the details of your work or cannot assess the relevance of your contribution. The conclusion section should be written as if it were a testament: as an expert in your subject, what is your message to the community? If your conclusion resembles your initial abstract, if it is written carelessly, your report will end without impact, as if it were a burden you want to get rid of. But if you worked hard during your internship, the conclusion is the finishing touch that will leave a good impression. As an indication, the conclusion should recall the path taken in your report, a summary of the results, perspectives, and especially your personal and thoughtful opinion.

      \subsection{Suggested Outline for the TER}
        For your TER report, the suggested outline is as follows:
        \begin{verbatim}\section{Introduction}
  \subsection{Group Presentation}
  \subsection{Topic Presentation}
\section{State of the Art}
\section{Work Done}
\section{Project Management}
\section{Conclusion}
\section{Perspectives and Personal Reflections}\end{verbatim}

    \section{Some Writing Guidelines}
      Of course, typos or language mistakes in reports give a very bad impression. You risk being judged poorly because of a spelling or syntax error. It is recommended to have your report proofread by a third party before submitting it. Also, from experience, reading a printed report provides more serenity and concentration than reading on a screen. Below are some additional tips to improve your writing style and avoid common mistakes found in student reports.

      \subsection{Typography}
        Each language has its own typographic rules that define how to space characters, highlight them, use punctuation, or choose fonts. French differs from English in several typographic rules; study them. Below are some non-exhaustive examples:
        \begin{itemize}
          \item French quotation marks are \og{} ... \fg{} and not `` ... '' as in English;
          \item When you choose a special font for certain titles or labels, keep exactly the same throughout your document. For example, if you decide to write function names like this \texttt{MyFunction} and variable names like this \texttt{\it MyVariable}, keep this identical form throughout your document; do not change from one section to another to avoid confusing the reader with form changes here and there. The same goes for software names, the font used for code, \etc
           \item Use bold characters sparingly, as they attract attention. Besides chapter, section, and subsection titles, \etc, you can use bold for the first definition of an important term or concept. Italics can also be used instead of bold;
          \item If you write your report in French, it is customary to put English or Latin terms in italics to distinguish them from French terms.
        \end{itemize}
      
      \subsection{References}
        Your report contains figures, tables, algorithms, sections, bibliographic references, \etc. These objects are referenced in your text with numbers (counters), for example: \og{} in table~2, we list the different cases... In figure~4.2, we illustrate the different steps... we will address this question in section~5\fg{}. Below are some tips regarding the use of references:
        \begin{itemize}
          \item For bibliographic references, a rule is that they can be removed without preventing the text from being read. For example, write: \og{} The PERT method~[4] allows ... \fg{} and not \og{} The method [4] allows ...\fg{};
          \item When an article has several authors, write the first author's name followed by the Latin abbreviation {\it et al.} For example: \og{} The article by Dupont {\it et al.} [14] described ... \fg{} and not \og{} The article by Pierre, Paul, Sara, and Jacques [4] described ...\fg{}, nor \og{} The article [4] described ...\fg{}. An exception is made for articles with only two authors, which can be cited as \og{} The article by Simon and Garfunkel [7] described ...\fg{};
          \item For all references, put a non-breaking space between the label and the number, as sometimes the editing tool separates the label from the number, which goes to the next line like this: \hfill\og{} ... Figure \\
            4 illustrates ... \fg{}. Avoid this unpleasant situation by putting a non-breaking space to force the editing tool not to separate the label and the following number. In \Latex, a non-breaking space is obtained with the tilde character \raisebox{0.75ex}{\texttildelow}.
        \end{itemize}
      
      \subsection{Format and Use of Colors for Figures, Graphs, Images, Photos, and Screenshots}
        In a document, the preferred format for figures, graphs, and images is vector format, not pixelated. The advantage of vector format is that it can be enlarged or reduced without pixelation or blurring, which is unfortunately very visible when printing or projecting. However, in some situations, such as photos and screenshots, the only possible format is pixelated. Make sure the resolution is sufficient in this case.

        Another important point is the use of colors. Prefer black and white (or grayscale) in all figures and diagrams for two main reasons:
        \begin{enumerate}
          \item  Think of colorblind readers who unfortunately cannot understand your colored document;
          \item Not all readers have access to a color printer; black and white printers are more common.
        \end{enumerate}
        
        So for your diagrams and graphs, make sure they are readable by everyone by using different types of lines (solid, dashed). To easily switch your document to grayscale, add \verb:\selectcolormodel{gray}: in your document preamble (before \verb:\begin{document}:).

    \section{Formal Writing of Ideas}
      Literary writing is a source of ambiguity and different interpretations from one reader to another. When you want to clarify your ideas, it is necessary to adopt formal writing. The universal precise language is mathematical language, which you should use if possible. On the one hand, this ensures that the reader can understand exactly what you describe. On the other hand, it allows another research team to reproduce your work, which is a fundamental virtue in scientific work (ideally, every scientific result can be reproduced by others). Thus, when necessary, it is good to formally define concepts, terms, and problems with mathematical writing. This allows them to be linked to lemmas, theorems, \etc.

      The same formal writing can be adopted for algorithms. Instead of writing code in a given programming language, it is better to write a clear algorithm, which can be coded in any programming language later, and which can also be analyzed more precisely (algorithmic complexity, proof of correctness, \etc).

  %%%%%%%%%%%%%%%%%%%%%%%%%
  % Examples              %
  %%%%%%%%%%%%%%%%%%%%%%%%%
  
  \pageblanche
  \part{Examples of \Latex Usage}

    In this part, some examples of \Latex usage are given.

    \section{Document Organization}

      A document is divided into sections; use the macro \verb"\section{}" to define a new section whose name is given in braces. A section can be divided into subsections with the macro \verb+\subsection{}+ which themselves can be divided into subsubsections \verb-\subsubsection{}-. If you want an additional level of division, you can use the macro \verb!\paragraph{}!.

      At the end of the introduction, it is always appreciated by the reader to have the organization of the document: Section xx presents the state of the art of blah blah blah. Then the work done is presented in section yy. References are explained in section \ref{sec:ref}.

      You can add a table of contents with the macro \verb#\tableofcontents#. Note that normally the name of the table of contents depends on the document language defined when loading the \verb'babel'~ package: \verb&\usepackage[english]{babel}&. The name can also be changed with the command: \verb@\renewcommand\contentsname{New name}@.

      The table of contents is usually at the very beginning of the document after an abstract.

      You can also add the list of figures and the list of tables with the macros \verb$\listoffigures$ and \verb=\listoftables=, respectively.

    \section{Header and Footer}
      Headers and footers have been modified using the \verb:fancyhdr: package.
      \begin{verbatim}
\fancyfoot{} % Empty footer

\fancyhead[LE]{\textsl{\leftmark}} % Header on the left on even pages
% (left even LE) current section name
\fancyhead[RE, LO]{\textbf{\thepage}} % Header on the right on even pages
% (right even RE) and on the left on odd pages (left odd LO) the current page number
\fancyhead[RO]{\textsl{\rightmark}} % Header on the right on odd pages
% (right odd RO) current subsection name
\end{verbatim}
      
    \section{Text Formatting}

      \subsection{Text Size}

        To change the font size, you can use the following macros:

        \begin{tabular}{ll}
          \verb|{\small small}| & {\small small}\\
          \verb|{\footnotesize smaller}| & {\footnotesize smaller}\\
          \verb|{\scriptsize even smaller}| & {\scriptsize even smaller}\\
          \verb|{\tiny very small}| & {\tiny very small}\\
          \verb|{\large large}| & {\large large}\\
          \verb|{\Large larger}| & {\Large larger}\\
          \verb|{\LARGE even larger}| & {\LARGE even larger}\\
          \verb|{\huge very large}| & {\huge very large}\\
          \verb|{\Huge very very large}| & {\Huge very very large}
        \end{tabular}
        
      \subsection{Font Shape, Weight, and Family}
        You can change the font shape, weight, or family using the following macros:

        \begin{tabular}{ll}
          \verb|{\bf bold}| or \verb|\textbf{...}| & {\bf bold} or \textbf{...}\\
          \verb|{\it italic}| or \verb|\textit{...}| & {\it italic} or \textit{...}\\
          \verb|{\sc small caps}| or \verb|\textsc{...}| & {\sc small caps} or \textsc{...}\\
          \verb|{\em emphasis}| or \verb|\emph{...}| & {\em emphasis} or \emph{...}\\
          \verb|{\tt typewriter}| or \verb|\texttt{...}| & {\tt typewriter} or \texttt{...}\\
          \verb|{\sf sans serif}| or \verb|\textsf{...}| & {\sf sans serif} or \textsf{...}
        \end{tabular}
        
      \subsection{Color}

        You can color text, in \textcolor{red}{red}, \textcolor{orange}{orange}, \textcolor{yellow}{yellow}, \textcolor{green}{green}, \textcolor{blue}{blue}, \etc. Many colors exist.

        You can also mix two colors \texttt{color1!percentage!color2}, the percentage corresponds to the percentage for the first color and the second color completes to 100. If no second color is specified, white is used. Text in \textcolor{green!40!black}{dark green}, \textcolor{green!60}{light green}, \textcolor{blue!50!red}{purple}.

        You can also define your own colors with the macro \verb|\definecolor|.

        \definecolor{blue}{rgb}{0, 0.6, 0.8}
        \definecolor{blue2}{RGB}{0, 153, 204}

        Text in \textcolor{blue}{blue} defined by rgb values between 0 and 1, and in \textcolor{blue2}{blue} defined by rgb values between 0 and 255, two colors defined by:

        \begin{verbatim}
\definecolor{blue}{rgb}{0, 0.6, 0.8}
\definecolor{blue2}{RGB}{0, 153, 204}\end{verbatim}

        \colorlet{red}{red}
        \colorlet{pine green}{green!40!black}

        You can also define colors from other colors, like \textcolor{red}{red} or \textcolor{pine green}{pine green} with the macro \verb|\colorlet|.
        \begin{verbatim}
\colorlet{red}{red}
\colorlet{pine green}{green!40!black}\end{verbatim}

    
    \section{Lists}

      \subsection{Itemized Lists}
        With \Latex you can make itemized lists with the \verb|itemize| environment:

        \begin{itemize}
          \item first item
          \item second item
          \begin{itemize}
            \item first sub-item
            \begin{itemize}
              \item Bla bla bla
              \item Bli bli bli
            \end{itemize}
            \item second sub-item
          \end{itemize}
          \item third item
        \end{itemize}
        
        You can change the bullet type for a specific list by adding the \verb|label| option to the itemize environment.

        \begin{itemize}[label = \ding{72}]
          \item first item
          \item second item
          \begin{itemize}[label = \textcolor{magenta}{\ding{95}}]
            \item first sub-item
            \begin{itemize}
              \item Bla bla bla
              \item Bli bli bli
            \end{itemize}
            \item second sub-item
          \end{itemize}
          \item third item
        \end{itemize}
        
        To change the bullet type for all itemized lists, simply redefine the macro \verb|\labelitemi| for the first level, \verb|\labelitemii| for the second, \verb|\labelitemiii| for the third, and \verb|\labelitemiiii| for the fourth.

        \begin{verbatim}
\renewcommand{\labelitemi}{\textcolor{blue}{\ding{43}}}\end{verbatim}

      \subsection{Ordered Lists}
        For ordered lists, use the \verb|enumerate| environment:

        \begin{enumerate}
          \item first item
          \item second item
          \begin{enumerate}
            \item first sub-item
            \begin{enumerate}
              \item Bla bla bla
              \item Bli bli bli
            \end{enumerate}
            \item second sub-item
          \end{enumerate}
          \item third item
        \end{enumerate}
        
        As with itemized lists, simply add the \verb|label| option to change the enumeration type for a specific list:

        \begin{enumerate}[label=\Roman* \ding{228}]
          \item first item
          \item second item
          \begin{enumerate}[label=\arabic*~:]
            \item first sub-item
            \begin{enumerate}[label=\alph*.]
              \item Bla bla bla
              \item Bli bli bli
            \end{enumerate}
            \item second sub-item
          \end{enumerate}
          \item third item
        \end{enumerate}
        
        \begin{enumerate}[label=\textcolor{magenta}{\Alph*)}]
          \item first item
          \item second item
          \begin{enumerate}[label=(\textcolor{blue}{\roman*})]
            \item first sub-item
            \begin{itemize}
              \item Bla bla bla
              \item Bli bli bli
            \end{itemize}
            \item second sub-item
          \end{enumerate}
          \item third item
        \end{enumerate}
        
        To make this change for all ordered lists, simply redefine the macro \verb|\labelenumi| for the first level, \verb|\labelenumii| for the second, \etc.

        \begin{verbatim}
\renewcommand{\labelenumi}{\Alph*)}\end{verbatim}
        
      \subsection{Descriptions}
        
        You can also make a descriptive list with the \verb|description| environment:

        \begin{description}
          \item[First] item
          \item[Second] item
          \item[Third] item
          \item[Long description] item
        \end{description}
        
    \section{Citations and References\label{sec:cite}}
      \subsection{Bibliographies}
        Bibliography management is one of the most appreciated features of \Latex. This is not done directly by the language, but by a third-party system: Bib\TeX.

        Bibliographic entries are stored in a \emph{.bib} file and inserted into the document using the \verb+\cite+ macro or the \verb|\citet| and \verb@\citep@ macros when using the \verb-natbib- package. It is thus very easy to make a bibliographic reference in the text, as for the article \citet{Lamport1986} or by citing it outside the sentence \citep{Lamport1986}. Note that it is possible to refer to several articles \citep{GoossensMS1994, GoossensRM1997, Klockl2000, Dongen2012}, to book chapters \citep[chap 2]{Gratzer2014} or to refer the reader \citep[see ][]{Datta2017}.

        References are displayed at the end of the document, automatically formatted according to the style you specified (for this document, it is the \emph{apalike-fr} style).

      \subsection{References\label{sec:ref}}
        In \Latex, the position of tables, graphs, \etc in the text generally does not matter since they are referred to using cross-references. Thus, these elements are placed in what are called floats, i.e., elements that are not anchored in the text but are moved during compilation to a place that seems most suitable (often the top of pages). So do not try in \Latex---as you probably do in a text editor---to place graphic elements immediately after the paragraph that refers to them, but let the system place them and refer to them explicitly using \verb|\label| and \verb|\ref|.

        Add \verb|\label{tag}| in the caption of the figure or table, \etc and \verb"\ref{tag}" where you refer to it. We are here in a subsection of section \ref{sec:cite}. We will talk more about tables and figures in section \ref{sec:tabfig}.

    \section{Theorems}
      The \verb!amsthm! package allows you to define styles for theorems. There are three styles:
      \begin{itemize}
        \item \verb|plain|: for theorems, lemmas, corollaries, conjectures
        \item \verb|definition|: for definitions, examples, problems
        \item \verb|remark|: for remarks, notes, conclusions
      \end{itemize}
      
      Before defining new theorems, you must set their style with the \verb|\theoremstyle{style}| macro.
      For example, define theorems for definitions and examples with the following lines:
      \begin{verbatim}
\theoremstyle{definition}
\newtheorem{definition}{Definition}
\newtheorem{exemple}{Example}\end{verbatim}

      And for remarks with the following lines:
      \begin{verbatim}
\theoremstyle{remark}
\newtheorem*{remarque}{Remark}\end{verbatim}

      \theoremstyle{definition}
      \newtheorem{definition}{Definition}
      \newtheorem{exemple}{Example}
      \theoremstyle{remark}
      \newtheorem*{remarque}{Remark}

      \begin{definition}[Graph]
        A \emph{graph} is a pair $G = (V, E)$ consisting of
        \begin{itemize}
          \item $V$ a set of vertices,
          \item and $E \subseteq \{\{x, y\}\ |\ (x, y) \in V^2 \wedge x \neq y\}$ a set of edges.
        \end{itemize}
      \end{definition}
      
      \begin{exemple}[Any Graph]
        Let $V = \{1, 2, 3\}$ and $E = \{\{1, 2\}, \{1, 3\}\}$. The pair $(V, E)$ is a \emph{graph}.
      \end{exemple}
      
      \begin{remarque}
        Note that definitions and examples have their own numbering and that the remark is not numbered because we used \verb|*| when defining remarks.
      \end{remarque}

    \section{Tables and Figures\label{sec:tabfig}}
      To insert figures in your document, you can use a png, jpg, eps, \etc image and include it with the \verb|\includegraphics| macro as for figure \ref{fig:logo}. You can also create it using the \verb|tikzpicture| environment as for figures \ref{fig:tikz} and \ref{fig:sort}. The manual \href{http://math.et.info.free.fr/TikZ/}{TikZ for the Impatient}, the site \href{https://texample.net/tikz/examples/}{TikZ examples} or \href{https://www.geogebra.org/}{Geogebra} can help you create the figure of your dreams.

      \begin{figure}
        \centering
        \includegraphics[width = 0.3\textwidth]{logo-master.png}
        \caption{A beautiful logo.\label{fig:logo}}
      \end{figure}
      
      \begin{figure}
        \centering
        \begin{tikzpicture}
          \node (logo) at (0,0) {\includegraphics[scale = 0.3]{logo-master.png}};
          \node [below = -0.2cm of logo] (master) {\Large\textsf{MASTER}};
          \node [below = -0.2cm of master] () {\Large\textcolor{blue!80!black}{\textsf{\textbf{COMPUTER SCIENCE}}}};
        \end{tikzpicture}
        \caption{Beautiful TikZ drawing \textcolor{magenta}{\ding{170}\ding{170}\ding{170}}.\label{fig:tikz}}
      \end{figure}
      
      \begin{figure}
        \centering
        \begin{tikzpicture}[scale = 0.75, rotate = -90, yscale = -1]
          \draw (0, 0) rectangle (5, 1);
          \foreach \x in {1, 2, ..., 5}
            \draw (\x, 0) -- (\x, 1);
            
          \draw (0.5, 0.5) node {1};
          \draw (1.5, 0.5) node {5};
          \draw (2.5, 0.5) node {3};
          \draw (3.5, 0.5) node {6};
          \draw (4.5, 0.5) node {4};
          
          \draw[magenta, thick] (2, 1) -- (2, 0);
          
          \draw[magenta, ->] (1, -0.2) -- (0, -1.8);
          \draw[magenta, ->] (3.5, -0.2) -- (4.5, -1.8);
          
          \begin{scope}[yshift=-0.5cm]
            \draw (-1, -2.5) rectangle (1, -1.5);
            \draw (3, -2.5) rectangle (6, -1.5);
            \foreach \x in {0, 4, 5}
              \draw (\x, -2.5) -- (\x, -1.5);
              
            \draw (-0.5, -2) node {1};
            \draw (0.5, -2) node {5};
            \draw (3.5, -2) node {3};
            \draw (4.5, -2) node {6};
            \draw (5.5, -2) node {4};
            
            \draw[magenta, thick] (0, -1.5) -- (0, -2.5);
            \draw[magenta, thick] (4, -1.5) -- (4, -2.5);
            
            \begin{scope}[yshift=-0.5cm]
              \draw[magenta, ->] (-0.5, -2.2) -- (-1, -3.8);
              \draw[magenta, ->] (0.5, -2.2) -- (1, -3.8);
              
              \draw (-1.5, -5) rectangle (-0.5, -4);
              \draw (0.5, -5) rectangle (1.5, -4);
              
              \draw (-1, -4.5) node {1};
              \draw (1, -4.5) node {5};
              
              \draw[magenta, ->] (3.5, -2.2) -- (3, -3.8);
              \draw[magenta, ->] (5, -2.2) -- (5.5, -3.8);
              
              \draw (2.5, -5) rectangle (3.5, -4);
              \draw (4.5, -5) rectangle (6.5, -4) (5.5, -5) -- (5.5, -4);
              
              \draw (3, -4.5) node {3};
              \draw (5, -4.5) node {6};
              \draw (6, -4.5) node {4};
              
              \draw[magenta, thick] (5.5, -5) -- (5.5, -4);
              
              \begin{scope}[yshift=-0.5cm]
                \draw[vert sapin, <-] (-0.5, -6.3) -- (-1, -4.7);
                \draw[vert sapin, <-] (0.5, -6.3) -- (1, -4.7);
                
                \draw (-1, -7.5) rectangle (1, -6.5) (0, -7.5) -- (0, -6.5);
                \draw (-0.5, -7) node {1};
                \draw (0.5, -7) node {5};
                
                \draw[magenta, ->] (5, -4.7) -- (4.5, -6.3);
                \draw[magenta, ->] (6, -4.7) -- (6.5, -6.3);
                
                \draw (4, -7.5) rectangle (5, -6.5);
                \draw (6, -7.5) rectangle (7, -6.5);
                
                \draw (4.5, -7) node {6};
                \draw (6.5, -7) node {4};
                
                \begin{scope}[yshift=-0.5cm]
                  \draw[vert sapin, <-] (5, -8.8) -- (4.5, -7.2);
                  \draw[vert sapin, <-] (6, -8.8) -- (6.5, -7.2);
                  
                  \draw (4.5, -10) rectangle (6.5, -9) (5.5, -10) -- (5.5, -9);
                  \draw (6, -9.5) node {6};
                  \draw (5, -9.5) node {4};
                  
                  \begin{scope}[yshift=-0.5cm]
                    \draw[vert sapin, <-] (3.5, -11.3) -- (3, -3.7);
                    \draw[vert sapin, <-] (5, -11.3) -- (5.5, -9.7);
                    \draw (3, -12.5) rectangle (6, -11.5) (4, -12.5) -- (4, -11.5) (5, -12.5) -- (5, -11.5);
                    
                    \draw (3.5, -12) node {3};
                    \draw (4.5, -12) node {4};
                    \draw (5.5, -12) node {6};
                    
                    \begin{scope}[yshift=-0.5cm]
                      \draw[vert sapin, <-] (1, -13.8) -- (0, -6.2);
                      \draw[vert sapin, <-] (3.5, -13.8) -- (4.5, -12.2);
                      
                      \draw (0, -15) rectangle (5, -14);
                      \foreach \x in {1, 2, ..., 5}
                        \draw (\x, -15) -- (\x, -14);
                      
                      \draw (0.5, -14.5) node {1};
                      \draw (1.5, -14.5) node {3};
                      \draw (2.5, -14.5) node {4};
                      \draw (3.5, -14.5) node {5};
                      \draw (4.5, -14.5) node {6};
                    \end{scope}
                  \end{scope}
                \end{scope}
              \end{scope}
            \end{scope}
          \end{scope}
        \end{tikzpicture}
        \caption{Tri par fusion dessiné avec tikz \textcolor{magenta}{\ding{170}\ding{170}\ding{170}}.\label{fig:sort}}
      \end{figure}
        
      To make nice tables, you can use the \verb|table| environment. A very simple example is given in table \ref{tab:ex}.

      \begin{table}
        \centering
        \begin{tabular}{l c r}
          \hline
          \multicolumn{3}{c}{Alignment}\\
          Left (\verb|l|) & Center (\verb|c|) & Right (\verb|r|)\\\hline
          Bla & Bla & Bla\\
          Riri & Fifi & Loulou\\
          Toto & Tata & Titi\\
          Left-aligned text & Centered text & Right-aligned text\\\hline
        \end{tabular}
        \caption{Table example.\label{tab:ex}}
      \end{table}
      
    \section{Mathematics}
      You can write very beautiful formulas with \Latex.
      % ...existing code for math formulas...
    \section{Algorithms}
      \begin{algorithm}
        \begin{algorithmic}
          \STATE tableau d'entiers tab \COMMENT{tableau d'entiers}
          \STATE int $i$ \COMMENT{indice de parcours}
          \STATE int $m$ \COMMENT{valeur maximale du tableau}
          \STATE
          \STATE $m \leftarrow$ tab[1]
          \FOR{$i$ \FROM 2 \TO length(tab)}
            \IF{$m <$ tab[$i$]}
              \STATE $m \leftarrow$ tab[$i$]
            \ENDIF
            \STATE \PRINT ``Le maximum est " + $m$
            \RETURN $m$
          \ENDFOR
        \end{algorithmic}
        \caption[Algorithme 1 (nom dans la liste des algorithmes)]{Met dans $m$ la valeur maximale du tableau tab.\label{ag:algo1}}
      \end{algorithm}
      
      L'algorithme \ref{ag:algo1} utilise le package \verb|algorithmic| dont la francisation des termes se trouve dans le fichier \verb+algo.sty+.
      
      \begin{algorithm}
        \begin{C}
int max(int* tab, int n) {
  int i; // indice de parcours
  int m; // valeur maximale du tableau
  
  m = tab[0];
  for (i = 1; i < n; i++) {
    if (m < tab[i]) {
      m = tab[i];
    }
  }
  
  printf("Le maximum est %d", m),
  return m;
}
        \end{C}
        \caption[Algo en C]{Retourne la valeur maximale du tableau tab.\label{ag:algoc}}
      \end{algorithm}
      
      \begin{algorithm}
        \begin{PseudoCode}
max(tableau d'entiers tab, entier n) {
  entier i // indice de parcours
  entier m // valeur maximale du tableau
  
  m <- tab[1]
  for i from 2 to n {
    if (m < tab[i]) {
      m <- tab[i]
    }
  }
  
  print("Le maximum est ", m),
  return m;
}
        \end{PseudoCode}
        \caption[Algo en PseudoCode]{Retourne la valeur maximale du tableau tab.\label{ag:algop}}
      \end{algorithm}
      
      \begin{algorithm}
        \begin{Java}
int max(int[] tab, int n) {
  int i; // indice de parcours
  int m; // valeur maximale du tableau
  
  m = tab[0];
  for (i = 1; i < n; i++) {
    if (m < tab[i]) {
      m = tab[i];
    }
  }
  
  System.out.println("Le maximum est " + m),
  return m;
}
        \end{Java}
        \caption[Algo en Java]{Retourne la valeur maximale du tableau tab.\label{ag:algoj}}
      \end{algorithm}
      
      Les algorithmes \ref{ag:algoc} en C, \ref{ag:algop} en pseudo code, et \ref{ag:algoj} en Java utilisent le package \verb|lstlistings|. La coloration syntaxique utilise le fichier \verb|colorationSyntaxique.sty| dans lequel sont définies les couleurs et les mot-clés. Vous pouvez modifier le fichier \verb|colorationSyntaxique.sty| pour ajouter de nouveaux mot-clés ou y ajouter un langage, pour le moment seuls C, Java, Python, Shell, R et un pseudo code sont disponibles.
      
    \section{Custom Commands}
      When you often use the same terms, macros, \etc, you can create new macros. For example, if you often use the \verb|\mathcal| macro, you can define a new macro:
      \begin{verbatim}
\newcommand{\mc}[1]{\ensuremath{\mathcal{#1}}}\end{verbatim}
        
      \newcommand{\mc}[1]{\ensuremath{\mathcal{#1}}}
      
      And then you can use it to write \mc{C} instead of $\mathcal{C}$.

      If you often use the term constraint programming, you can define a macro for it:
      \begin{verbatim}
\newcommand{\cp}{constraint programming\xspace}\end{verbatim}

      \newcommand{\cp}{constraint programming\xspace}

      So every time I want to write \cp, I just use the \verb|\cp| macro, and that's quite nice, just like \cp.

    \pageblanche
    \appendix
    
    \section{More on Bibliography}
      In terms of form, we can classify scientific documents into three categories, ranked in order of importance as follows:
      \begin{enumerate}
        \item Documents reviewed by a committee of experts: these are articles or theses published after independent and competent readers have reviewed the document. This is the case, for example, for doctoral theses, articles in scientific journals, or conferences with a review committee;
        \item Documents published but not reviewed by a committee of experts: these are documents that have undergone an editing process for form, but whose scientific content has not been reviewed by an independent committee. This is the case, for example, for scientific books;
        \item Documents not reviewed by independent experts: these are documents that only engage the authors. This is the case, for example, for research reports, technical reports, \etc.
      \end{enumerate}
      
      In terms of quality, we can classify scientific documents into seven categories without particular hierarchy:
      \begin{enumerate}
        \item  Fundamental document: it brings real added value to the field, a scientific novelty not known until its publication, on which other work will be based;
        \item  Formal document: a document written in mathematical language,
