\documentclass{rapport}
\usepackage[utf8]{inputenc}
\usepackage[T1]{fontenc}
\usepackage[english]{babel}

\usepackage{pifont}
\usepackage{url}
\usepackage[colorlinks, citecolor=red!60!green, linkcolor=blue!60!green, urlcolor=magenta]{hyperref}

\usepackage{algorithm}
\usepackage{algo}
\usepackage{colorationSyntaxique}

\pagestyle{fancy}
\renewcommand{\sectionmark}[1]{\markboth{\thesection.\ #1}{}}
\fancyfoot{}
\fancyhead[LE]{\textsl{\leftmark}}
\fancyhead[RE, LO]{\textbf{\thepage}}
\fancyhead[RO]{\textsl{\rightmark}}

\def\Latex{\LaTeX\xspace}
\def\etc{\textit{etc.}\xspace}

\title{Project Report Title}
\author{Alice Example, Bob Example}
\supervisor{Prof. Smith, Dr. Lee}
\date{Spring Semester 2025--2026}

% Optional metadata (override defaults from rapport.cls)
\universityname{University Name}
\type{Project}
\formation{Master's Program in Computer Science}

\begin{document}

\maketitle

\begin{abstract}
This document is an English starter template for student reports.
It shows a clean default structure and basic \Latex usage.
For more documentation, see \url{https://www.latex-project.org/help/documentation/}
and \url{https://tex.stackexchange.com/}.
\end{abstract}

\clearpage
\tableofcontents

\clearpage
\part{Guidelines and Structure}

\section{Introduction}
Present the context, scope, and motivation of your project.
State the problem, the expected outcome, and the main contributions.

\section{Related Work}
Summarize prior work relevant to your topic.
Explain how your approach differs from existing methods.

\section{Methodology}
Describe your design choices, architecture, and implementation details.
Use figures and tables to support key technical decisions.

\section{Results and Discussion}
Report experiments, metrics, and observations.
Discuss limitations and potential improvements.

\section{Conclusion}
Recap the main results and list clear next steps.

\part{Examples}

\section{Cross-references}
You can reference sections, figures, and tables throughout the report.
For example, see Section~\ref{sec:algo} for a simple algorithm example.

\section{Algorithm Example\label{sec:algo}}
\begin{algorithm}[h]
\caption{Linear Search}
\begin{algorithmic}[1]
\FORALL{$x$ in list}
  \IF{$x = target$}
    \RETURN true
  \ENDIF
\ENDFOR
\RETURN false
\end{algorithmic}
\end{algorithm}

\section{Table Example}
\begin{table}[h]
\centering
\begin{tabular}{|l|c|r|}\hline
Left aligned & Centered & Right aligned \\\hline
Item A & 10 & 0.95 \\\hline
Item B & 12 & 0.97 \\\hline
\end{tabular}
\caption{Example table.}
\label{tab:example}
\end{table}

\section{Citation Example}
Citations can be inserted with \verb|\cite|, \verb|\citet|, and \verb|\citep|,
for example \citet{Lamport1986}.

\clearpage
\bibliographystyle{apalike}
\bibliography{biblio}

\end{document}
